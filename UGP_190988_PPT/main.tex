\documentclass [xcolor=svgnames, t] {beamer}
\usepackage[utf8]{inputenc}
\usepackage{booktabs, comment}
\usepackage[absolute, overlay]{textpos}
\usepackage{pgfpages}
\usepackage[font=footnotesize]{caption}
\useoutertheme{infolines}

\setbeamertemplate{page number in head/foot}{}
\usepackage{csquotes}


\usepackage{amsmath}
\usepackage[makeroom]{cancel}


\usepackage{textpos}

\usepackage{tikz}

\usetheme{Madrid}
\usepackage{tikz}



\title[Chatterjee's Correlation in MCMC]{Applications of Chatterjee's Correlation in MCMC}
\subtitle{(UGP, 21-22 Even Semester)}
\institute[IITK]{Department of Mathematics and Statistics \\Indian Institute of Technology, Kanpur}
\titlegraphic{\includegraphics[height=2.0cm]{iitk_logo.jpg}}
\author[Vivek Kumar Singh]{
	Vivek Kumar Singh,
	Dootika Vats }


\institute[]{Department of Mathematics and Statistics \\Indian Institute of Technology, Kanpur}
\date{\today}


\addtobeamertemplate{navigation symbols}{}{%
    \usebeamerfont{footline}%
    \usebeamercolor[fg]{footline}%
    \hspace{1em}%
    \insertframenumber/\inserttotalframenumber
}

\begin{document}
\begin{frame}
\maketitle
\end{frame}


%%%%%%%%%%%%%%%%%%%%%%%%%%%%
\logo{\includegraphics[height=1.0cm]{iitk_logo.jpg}~%
}
%%%%%%%%%%%%%%%%%%%%%%%%%%



\begin{frame}
\frametitle{Table of Contents}
\tableofcontents
\end{frame}

\section{Markov chain Monte Carlo}
\begin{frame}
    \frametitle{Markov Chain Monte Carlo}
    \begin{itemize}
        \item A Markov chain is a discrete time stochastic process $X_1, X_2, \dots$ such that the next state of the process depends solely on the present state, i.e.
            $$\Pr(X_n | X_1, \dots, X_{n-1}) = \Pr(X_n | X_{n-1}).$$
            \vspace{2em}
        \item A Markov chain can be specified by two things
            \begin{enumerate}
                \item The initial distribution, i.e. the marginal of $X_1$.
                \item The transition probabilites, i.e. the conditional of $X_{n+1}|X_n$
            \end{enumerate}
    \end{itemize}
\end{frame}

\begin{frame}
    \frametitle{Markov chain Monte Carlo}
    \begin{itemize}
        \item A Markov chain is said to be \textbf{stationary} if the marginal of $X_n$ is independent of $n$. This invariant distribution is called the stationary distribution.
        \vspace{2em}
        \item A Markov chain is \textbf{ergodic} if the distribution of $X_n$ converges to the invariant distribution.
        \vspace{2em}
        \item A stationary Markov chain is \textbf{time-reversible} w.r.t. the stationary distribution if $X_n$ and $X_{n+1}$ are exchangable.
    \end{itemize}

\end{frame}

\section{Pearson's Correlation Coefficient}
\begin{frame}
    \frametitle{Pearson's Correlation Coefficient}


\end{frame}

\section{Chatterjee's Correlation Coefficient}

\section{Chatterjee's Autocorrelation Coefficient}

\section{Sketch of proof of consistency}

\section{Simulations}


\begin{frame} [allowframebreaks]\frametitle{References}
        \bibliographystyle{apalike}
        \bibliography{bibfile}
\end{frame}

\end{document}