\chapter{Problems with Pearson correlation coefficent}
\begin{definition}
	$\textbf{Pearson correlation coefficient}$ measures the linear correlation of two sets of data.
	Given a pair of random variables $(X, Y)$, pearson correlation $\rho$ is defined as
	$$\rho = \frac{\Cov(X, Y)}{\sqrt{\Var{X}} \cdot \sqrt{\Var{Y}}}$$
\end{definition}

Pearson's correlation coefficent is very powerful in detecting monotone relations and
has a well developed asymptotic theory. The autocorrelation function that we look at in MCMC is also defined using this.
\newline

There are two most common problems with this coefficent.
\begin{enumerate}
	\item Firstly, we would like that the correlation would be close to its maximum value
	if and only if one variable looks like a noiseless function of the other variable.
	This is not the case for the Pearson's Coefficient as it is close to $\pm 1$ iff one variable is a noiseless $\textit{linear}$ function.
	\item Second, we would like the correlation to be close to its minimum value if and only if both the variables are independent of each other.
	In the case of the Pearson's correlation, it is zero when the variables are independent but the converse is not always true.
\end{enumerate}
