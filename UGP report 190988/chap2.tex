\chapter{Preliminaries}
\begin{definition}
	$\textbf{(Time-homogeneous markov chain).}$ The $\mathbb{R}$-valued sequence of random variables $X_1, X_2, \dots$ is a time-homogeneous Markov chain if for all $A \in \mathcal{B}(\mathbb{R})$ and for all $n \in \mathbb{N}$
	$$\Pr(X_{n+1} \in A | X_1, X_2, \dots, X_n) = \Pr(X_{n+1} \in A | X_n)$$
\end{definition}
\begin{definition}
	$\textbf{(Markov Transition Kernel).}$ A Markov transition kernel is a map $P: \mathbb{R} \times \mathcal{B}(\mathbb{R}) \rightarrow [0, 1]$ such that
	\begin{enumerate}
		\item for all $A \in \mathcal{B}(\mathbb{R}), P(. A)$ is a measurable on $\mathbb{R}$.
		\item for all $x \in \mathbb{R}, P(x, .)$ is a probability measure on $\mathcal{B}(\mathbb{R})$.
	\end{enumerate}
\end{definition}
\begin{definition}
	$\textbf{(Stationarity):}$ A discrete-time Markov chain $X_1, X_2, \dots$ is stationary if the distribution of $X_n$ does not depend on $n$.
\end{definition}
\begin{definition}
	\textbf{(Time Reversibility).}
\end{definition}
\begin{definition}
	\textbf{(Total Variation Norm).}
\end{definition}
\begin{definition}
	\textbf{(Ergodicity).}
\end{definition}
\begin{definition}
	\textbf{(Chapman-Kolmogorov Equation).}
\end{definition}
\begin{definition}
	\textbf{(Glivenko-Cantelli Theorem).}
\end{definition}
\begin{theorem}
	\textbf{(Lebesgue's Dominated Convergence Theorem).}
\end{theorem}
