\chapter{Preliminaries}
\section{Introduction to Markov chain Monte Carlo}
\begin{definition}
	(Time-homogeneous markov chain). The $\mathbb{R}$-valued sequence of random variables $X_1, X_2, \dots$ is a time-homogeneous Markov chain if for all $A \in \mathcal{B}(\mathbb{R})$ and for all $n \in \mathbb{N}$
	$$\Pr(X_{n+1} \in A | X_1, X_2, \dots, X_n) = \Pr(X_{n+1} \in A | X_n)$$
\end{definition}
\begin{definition}
	(Markov Transition Kernel). A Markov transition kernel is a map $P: \mathbb{R} \times \mathcal{B}(\mathbb{R}) \rightarrow [0, 1]$ such that
	\begin{enumerate}
		\item for all $A \in \mathcal{B}(\mathbb{R}), P(. A)$ is a measurable on $\mathbb{R}$.
		\item for all $x \in \mathbb{R}, P(x, .)$ is a probability measure on $\mathcal{B}(\mathbb{R})$.
	\end{enumerate}
\end{definition}
\begin{definition}
	stationarity
\end{definition}
\begin{definition}
	ergodicity
\end{definition}
\begin{definition}
	time reversibility
\end{definition}
\begin{definition}
	total variation norm
\end{definition}
\begin{definition}
	chapman Kolmogorov
\end{definition}

\section{Basic Theorems from Measure Theory}
\begin{theorem}
	Lebesgue's DCT
\end{theorem}
