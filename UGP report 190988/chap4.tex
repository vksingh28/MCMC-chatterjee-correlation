\chapter{Chatterjee's autocorrelation function}

Sourav Chatterjee proposed a correlation coefficient in his [add reference].
This coefficient is (a) as simple as the classical ones, (b) is a consistent estimator of some measure of dependence which is 0 iff the variables are independent, and 1 iff one is a measurable function of the other, and (c) has a simple asymptotic theory under the hypothesis of independence, like the classical coefficients. \\\\
Let $(X, Y)$ be a pair of random variables, where Y is not a constant (for our purposes, both X and Y are continuous). Let $\{(X_i, Y_i)\}_{i = 1}^{n}$ be i.i.d. pairs following the same distribution as $(X, Y)$.
\begin{enumerate}
    \item The case when $X_i's \text{ and } Y_i's$ have no ties. Rearrange the data as\\ $(X_{(1)}, Y_{(1)}), \dots, (X_{(n)}, Y_{(n)})$ such that $X_{(1)} < \dots < X_{(n)}$. Let $r_i$ be the rank of $Y_{(i)}$, i.e. the number of $j$ such that $Y_{(j)} \leq Y_{(i)}$.  Then the correlation coefficient $\xi_n$ is defined to be
    $$\xi_n(X, Y) := 1-\frac{3\sum_{i=1}^{n-1} |r_{i+1} - r_i|}{n^2-1}$$.

    \item In the case of ties. If there are ties in $X_i's$, choose an increasing arrangement as follows and break ties uniformly at random. Let $r_i$ defined as above, and define $l_i$ to be the number of $j$ such that $Y_{(j)} \geq Y_{(i)}$. Define
    $$\xi_n(X, Y) := 1-\frac{n\sum_{i=1}^{n-1} |r_{i+1} - r_i|}{2\sum_{i=1}^{n-1}l_i(n-l_i)}$$.
    When there are no ties among the $Y_i's, l_1, \dots, l_n$ is just a permutation of $1, \dots, n$ and the denominator is just $n(n^2-1)/3$, which reduces to the definition in the no ties case.
\end{enumerate}

\begin{theorem}
If $Y$ is not almost surely a constant, then as $n \rightarrow \infty$, $\xi_n(X, Y)$ converges almost surely to the deterministic limit
$$\xi(X, Y) := \frac{\int Var(\mathbb{E}(1_{\{Y \geq t\}}|X)) d\mu(t)}{\int Var(1_{\{Y \geq t\}}) d\mu(t)}$$
where $\mu$ is the pdf of $Y$. This limit belongs to the interval $[0, 1]$. It is 0 iff X and Y are independent, and it is 1 iff there is a measurable function $f:\mathbb{R} \rightarrow \mathbb{R}$ such that $Y = f(X)$ almost surely.
\end{theorem}
