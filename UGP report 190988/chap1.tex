\chapter{Introduction}

Markov chain Monte Carlo (MCMC) methods are a class of algorithms used for sampling from complicated probability distributions.
They are often required for parameter estimation in the statistical models encountered in real-world applications.

If $X_1, X_2, \dots$ is the Markov chain,
then the lag-k autocorrelation is defined as
$$\gamma_k = \rho(X_1, X_{1+k}),$$
where $\rho$ is the Pearson's correlation coefficent (See Section []).
The autocorrelation function is used for assessing the quality of the Markov chain produced.

The Pearson's correlation is great at detecting monotone relations in data,
but fails when applied to non-linear situations (say $Y = \sin(X)$)

Chatterjee proposed a new measure of dependence in [site paper] which is
(a) as simple as the Pearson correlation,
(b) is a consistent estimator of some measure of dependence which is 0 if and only if the variables are independent and 1 if and only if one is a measurable function of the other, and
(c) has a simple asymptotic theory under the hypothesis of independence, like the Pearson correlation.
See Section [] for more on this.

The advantages of this new measure motivated us to define a new autocorrelation function using the Chatterjee's correlation coefficent instead of Pearson's.
In order to use this in MCMC theory, two things are needed,

1. Some properties that are followed by the classical acf should also hold true for our version for Markov chains (See Section []).

2. As we don't have the luxury of i.i.d. draws, for which the consistency of the estimator of Chatterjee's correlation holds (See []),
	we need to prove it for the case of MCMC samples we have (See Section []).

We were able to prove three new results related to the Chatterjee's autocorrelation that also hold true for the classical acf.
We also believe that the estimator for Chatterjee's correlation coefficent is consistent even when we're using correlated draws from a stationary Markov chain.
We were able to change some parts of the proof of consistency presented in [og paper], and believe that the parts left can also be done and are left as future work.